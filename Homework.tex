\documentclass{article}

\usepackage{mathtools}  % useful for paired delimiters

\title{The Pigeonhole Principle}
\author{Kareem Khalifa}
\date{August 23, 2023}

% Paired delimiter
\DeclarePairedDelimiter{\floor}{\lfloor}{\rfloor}
\DeclarePairedDelimiter{\ceiling}{\lceil}{\rceil}


\begin{document}
\maketitle
\
The Pigeonhole Principle says that when you try to place a greater number of items into fewer compartments, at least one of those compartments must hold more than one item. This principle draws an analogy or you could say silmile as its comparing having more pigeons (objects) than pigeonholes (containers), indicating that some pigeonholes must inevitably have more than a single pigeon.

The Extended Pigeonhole Principle expands upon this concept by taking into account fractional parts of the object count being distributed. It declares that when you distribute n objects into k containers, you can expect that at least one container will contain n/k objects.

Hello, World!

This is an inline equation: $y = mx + b$

This is a fraction: \( \frac{1234}{3456} \)

This is a set \[ = \{0, 1, 2, 3, 4, 5, 6, 7,\} \]

This is a ceiling: \[ [lceil \frac{111}{5} \rceil \]
\newpage
This is a better ceiling: \[ \ceiling*{\frac{111}{5}} \]

\( |S| \)

\begin{itemize}
\item Broccoli
\item Beans
\item Bluebell Icecream
\item Burger Buns
\item Beef
\item Bowling Ball
\end{itemize}

\begin{enumerate}
\item {Denzel Curry}
\item {SZA}
\item {Aubrey Graham}
\item {Huachengyu}
\end{enumerate}

Floor(x) = Largest Integer $\le$ x

$ \floor*{\frac{1,000,017}{1,000,019}} \approx \floor{0.999} = $  



$S = \{2, 4, 6, \ldots, 50,000\} $

\end{document}
